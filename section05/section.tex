\section{Testing}
\label{sec:Testing}
% Section: Testing
% Describe any testing performed on your code. (this will likely be a short chapter as you did not complete any formal testing I believe).
Agile testing focuses on repairing faults immediately, rather than waiting for the end of the project. So in Agile development, testing is conducted during each iteration after its implementation phase. In each testing, we followed the sequences listed below to test:

\subsection{Functional Testing}
The first step of web application testing ensures that the functions of a system are tested. During functional testing, actual system usage is simulated. The idea is to come as close as possible to real system usage and create test conditions that are related to user requirements.

Based on the black-box testing, we analyzed the boundaries of each system unit and created corresponding test cases. The specific implementation is to make legal or illegal input for each form and analyzed the expected output with the actual output.

\subsection{Usability Testing}
Usability goes beyond functionality testing and combines testing for functionality as well as overall user experience. Because the developer knows how the code is written, he may not think to test a specific scenario. So we recruit test participants to use the application, based on their users' perspectives, we have received a lot of feedback from all aspects. For example, they don't know how to use the map editor correctly, etc.

\subsection{Interface Testing}
Interface testing ensures that all interactions between the client and web server interfaces are running smoothly. We tested each of the APIs, including checking the communication process and ensuring that error messages are correctly displayed and that client and server interrupt are handled correctly. There are also a lot of problems, such as executing database operations without further verification of the parameters obtained on the server side, etc.

\subsection{Compatibility Testing}
