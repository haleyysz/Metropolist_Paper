\section{Conclusion}
\label{sec:Conclusion}

With this current project, we can log in as an administrator or register as a system user and log in. Both administrators and users can edit their profile and log out. In particular, administrators can view all users, disable or enable normal users, and search for a user. While users can create their own maps and edit or delete them. Users can also view their all maps, search for maps, make maps public or private, and download maps.

There are still some requirements that have not been implemented, such as view and search for users, make maps viewable by only a select list of users, make maps viewable only by a select list of users, and annotate a map including its sub-requirements. We hope to complete them in future work.

We inevitably encountered some difficulties in the process of development, mainly in the development part of the city map generator. At the beginning of the choice of third-party libraries, we tried a few, which are not very ideal. Next, on the choice of semi-automatic map generation algorithm, we tried the force-directed graph layout \cite{web:D3-force}. The purpose is to allow buildings to locate according to the polygons' boundaries automatically, and we made several good demos. However, in the end, we found that the requirements could not be completed, and even slightly deviated from the original ones. We finally found the correct algorithm that is the polygon partitioning algorithm.

We would consider using SVG as the rendering engine because we decided on using Canvas in the beginning, and it is increasingly difficult to replace it as the number of code increased and the project grew. Also, it is undeniable that the rendering capability of Canvas began to decline as the number of polygons in the local map has increased to a certain amount or zoomed in the map to a certain scale. So we want to try to develop it with SVG and hope that its rendering capability can support this project.

From the software engineering perspective, we tried following agile principles as much as possible. However, in fact, in the entire development process, we have lost the role of the customer, basically replaced by the advisor, that is, the sponsor. In a sense, we pretended to be agile, although the result of the final project is acceptable, there are still many differences with the actual development process. But if we start over the project and choose the Software Development Life Cycle Model again, we will still choose Agile, because even if we choose other development models, we still lack the role of the customer. More importantly, due to the uncertainty of the various selections, we need to keep trying. So the development process must be flexible, and Agile can support this very well.
